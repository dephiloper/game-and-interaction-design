\documentclass[11pt]{scrartcl}
\usepackage[utf8]{inputenc}
\usepackage[german]{babel}
\usepackage{geometry}
\usepackage{hyperref}
\usepackage{listings}
\newcommand{\lbparagraph}[1]{\paragraph*{#1}\mbox{}\\}
\geometry{
  left=3cm,
  right=3cm,
  top=3cm,
  bottom=3cm,
  bindingoffset=5mm
}
\usepackage{graphicx}
\title{Projektdokumentation GT2}
\subtitle{\vspace{2mm} Game \& Interaction Design}
\author{Jannik Schmitz,\\Maximilian Gertz,\\Bruno Schilling,\\Philipp Bönsch}
\date{\today}

\begin{document}
\maketitle
\lstset{basicstyle=\ttfamily\small,breaklines=true}
\tableofcontents
\newpage
\section{Einleitung}
\section{Konzept}
\section{Hauptaspekte des Spiels}
\subsection{Allgemeine Spielbeschreibung}
\subsection{Elemental Tetrad}
\subsubsection{Experience}
\subsubsection{Fantasy}
\subsubsection{Mechanics}
\subsubsection{Mechanics}
\lbparagraph{Actions}
Lorem ipsum d
\lbparagraph{Rules}
Lorem ipsum do
\lbparagraph{Goals}
Lorem ipsum dolar
\subsubsection{Aesthetics}
\subsubsection{Technology}
\section{Entwicklungsprozess}
\subsection{Ideenfindung}
\lbparagraph{Omega}
Lorem ipsum d\cite{S2014}
\lbparagraph{Attack on Human}
Lorem ipsum do
\lbparagraph{Gravity Warrior}
Lorem ipsum dolar
\subsection{Erster Prototyp des Player Movement}
\subsection{Das Konzept des Spiels}
\subsection{Erweiterung der Features}
\lbparagraph{Weitere Alienarten}
Lorem ipsum d
\lbparagraph{Buff-Selection}
Lorem ipsum do
\lbparagraph{Waffentypen}
Lorem ipsum dolar
\lbparagraph{Eigenschaften Asteroiden}
Lorem ipsum dolar
\lbparagraph{Reduktion der Features}
Lorem ipsum dolar
\subsection{Abstimmung der Spielelemente}
\subsection{Verbesserung der Experience}
\section{Fazit}
\subsection{Erweiterungsmöglichkeiten}
\section{Anhang}
\subsection{User Tests}
\subsection{Repository}
\bibliographystyle{plain}
\bibliography{references}
\end{document}