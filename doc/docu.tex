\documentclass[11pt]{scrartcl}
\usepackage[utf8]{inputenc}
\usepackage[german]{babel}
\usepackage{geometry}
\usepackage{hyperref}
\usepackage{listings}
\usepackage{subcaption}
\newcommand{\lbparagraph}[1]{\paragraph*{#1}\mbox{}\\}
\renewcommand{\baselinestretch}{1.2}
\geometry{
  left=3cm,
  right=3cm,
  top=3cm,
  bottom=3cm,
  bindingoffset=5mm
}
\usepackage{graphicx}

\begin{document}

\begin{titlepage}
	\centering
	{\scshape\LARGE Hochschule für Technik und Wirtschaft Berlin \par}
	\vspace{1cm}
	{\Huge \scshape{Dokumentation\\ Gravity Warrior}\par}
	\vspace{1cm}
	{\scshape\Large GT2 - Game and Interaction Design\par}
	\vspace{2cm}
	\includegraphics[width=0.9\linewidth]{title.jpg}\par\vspace{1cm}
	\vspace{1cm}
	{\large\itshape Jannik Schmitz, Maximilian Gertz,\\Bruno Schilling, Philipp Bönsch\par}
	\vfill
	
% Bottom of the page
	{\large \today\par}
\end{titlepage}

\lstset{basicstyle=\ttfamily\small,breaklines=true}
\newpage
\tableofcontents
\newpage
\section{Einleitung}
Im Rahmen der Veranstaltung GT2 Game \& Interaction Design wurde die Aufgabe gestellt, kursbegleitend ein prototypisches Spiel zu entwickeln. Dazu war es initial notwendig, dieses konzeptuell zu designen und anschließend zyklisch den Stand der Entwicklung in Form von Präsentationen vorzustellen. Für die Erarbeitung des Spiels stand den Teilnehmern der Veranstaltungen ein Zeitraum von etwa 3 Monaten zur Verfügung. Der Hauptaspekt des Kurses lag darin, einen genauen Überblick über die einzelnen Kernbereiche des Game Design zu vermitteln, um diese anschließend in der Umsetzung des eigenen Prototyps zu adaptieren. Besonders das Elemental Tetrad\cite[~p.41]{S2014} stand dabei im Zentrum des Entwicklungsprozesses, wobei sich die Teilnehmer des Kurses genauer mit den Bereichen Mechanics, Technology, Aesthetics, Fantasy und der Experience des Spiels befassen sollten.
Diese Dokumentation beschreibt den Entwicklungsprozess, die angewandten Verfahren im Bereich Game \& Interaction Design sowie Hürden und Lösungsansätze, welche während der Umsetzung des Spiels aufgekommen sind. Das Projektteam bestand aus vier Mitgliedern, wobei die Bereiche Programmierung, Illustration (Assets) und Gamekonzept von allen Mitgliedern ausgeübt wurden. Später erfolgte jedoch eine Spezialisierung, wobei sich die Hälfte auf die Entwicklung im Bereich Programmierung und die andere Hälfte auf das Erstellen von Illustrationen und Sounds fokussierte. Dies kam zum einem aufgrund der Game Engine Godot (siehe Technology) zustande, welche für zwei Mitglieder erst eine gewisse Eingewöhnung abverlangte und zum anderen lag es an der Tatsache, dass die Entwicklung des Spiels zielgerichteter und schneller vorangehen sollte.

\newpage
\section{Konzept}
Es wurden insgesamt drei verschiedene Spiel Konzepte ausgearbeitet, welche mit dem Professor besprochen und evaluiert wurden. Daraufhin wurde sich auf das Konzept des Gravity Warriors geeinigt, da dieses inhaltlich am meisten überzeugt hat. Im Folgenden wird deshalb lediglich das Gravity Warrior Konzept vorgestellt. 

Das wichtigste Ziel des Gravity Warrior Konzepts war ein 2D gravitationsbasierter Shooter in der Sideview-Optik zu entwickeln. Dabei lag der Fokus des Spiels auf der Gravitation, die von verschiedenen Planeten (später Asteroiden) ausging. Deshalb waren Projektile geplant, die von der Gravitation beeinflusst werden (siehe \autoref{fig:concept}: \nameref{fig:concept}). Zusätzlich sollte ein Wavesystem implementiert werden. Die Spieler sollten zunächst schemenhaft als Quadrat und die Gegner als Dreiecke dargestellt werden, was skizzenhaft in \autoref{fig:concept}: \nameref{fig:concept} dargestellt wird.

\begin{figure}[htp]
	\centering
	\includegraphics[width=.3\textwidth]{sketch_01.png}\hfill
	\includegraphics[width=.3\textwidth]{sketch_02.png}\hfill
	\includegraphics[width=.3\textwidth]{sketch_03.png}
	\caption{Konzeptskizze}
	\label{fig:concept}
\end{figure}

Zudem war ein schwarzes Loch im Universum geplant, in das der Spieler nicht fallen darf. Dieses schwarze Loch wird an einem zufälligen Ort platziert. Darüber hinaus sollten weitere gefährliche Umgebungen integriert werden. Des Weiteren sollten mehrere Spielmodi angeboten werden: Battle Royale, Competitive, Boss Fights, Coop und Hardcore. Im Rahmen des ersten Konzeptes wurde ein Moodboard erstellt, das die Farben und die einzelnen Objekte im groben skizziert (siehe \nameref{sec:appendix}).

\newpage
\section{Hauptaspekte des Spiels}

\subsection{Allgemeine Spielbeschreibung}
Gravity Warrior ist ein 2D Couch-Coop Twin-Stick Shooter, bei dem der Spieler einen von vier Charakteren steuert (Gravity Warrior genannt), der mit einem Jet Pack und einer Plasmapistole ausgestattet ist. Neben der Steuerung der Warrior und dem Abfeuern der Plasmapistole ist die Gravitation die Kernmechanik des Spiels. Diese wird ausgehend von den Asteroiden im Spiel erzeugt und beeinflusst dabei sowohl den Spieler als auch dessen Pistolenprojektile. Nach einem kurzen Vertrautmachen mit der Steuerung werden der Spieler und seine Mitstreiter von Wellen gegnerischer Aliens angegriffen. Anhand der Story wird dem Spieler vermittelt, dass dieser Teil eines Forschungsteams ist und, dass das Team aufgrund einer zu hohen Gravitation auf der aktuell befindlichen Asteroidenkonstellation gestrandet ist. Hilfe kann ausschließlich durch einen Satelliten gerufen werden, welcher im Laufe des Spiels ebenfalls von den anbahnenden Gegnerscharen attackiert wird. Nun müssen sich die Gravity Warrior gegen unzählige Aliens behaupten, deren Angriffen ausweichen und die Energieanzeige des Satelliten überwachen um zum Heimatplaneten zurückzukehren.

Die Steuerung eines Gravity Warriors erfolgt über ein Gamepad, wobei der linke Joystick für die Bewegung des Spielers und der rechte Joystick für das Zielen mit der Waffe benutzt werden. Da diese Art der Steuerung als eher fordernd für Spieler eingeschätzt wurde, ist bei der Umsetzung des Spiels darauf geachtet worden, dass der Spieler nicht mit zu komplexen Aufgaben im Spiel überfordert wird. Als Setting des Spiels wurde eine düstere und leicht unheimliche Weltraumatmosphäre gewählt. Zusätzlich wurde ein minimalistischer Grafikstil mit wenigen Farbakzenten verwendet, um den Spieler auf die wichtigen Bereiche des Spiels wie näher kommende Gegner und fliegende Projektile aufmerksam zu machen. Aufgrund der verschiedenen Schwierigkeitseinstellungen, welche die Spieler zu Beginn des Spiels auswählen können, ist die Dauer eines Matches nicht auf eine definierte Zeit festgelegt. Im Standardschwierigkeitsgrad ist das Spiel jedoch durchschnittlich in 10 Minuten bestritten.


\subsection{Elemental Tetrad}
Laut dem Konzept Elemental Tetrad von Schell\cite[~p.41]{S2014} ist es sinnvoll, die verschiedenen Aspekte des Game Designs in unterschiedliche Kategorien zu unterteilen. Diese Kategorien sind Fantasy bzw. Story, Mechanics, Aesthetics und Technology. Daraufhin werden die Aspekte innerhalb der Kategorien bezüglich der Spielererfahrung (Experience) analysiert.

Deshalb wird in dem nachfolgenden \autoref{subsec:exp} die \nameref{subsec:exp} dargestellt. Darauf folgend wird in \ref{subsec:fan} bis \ref{subsec:tech} auf die oben genannten Kategorien eingegangen.

\subsubsection{Experience}
\label{subsec:exp}
Laut Schell ist die Experience aus der Sicht des Spielers der ideale Startpunkt, um ein Spiel zu designen \cite[~p. xxiii]{S2014}. Deshalb wird im Folgenden die Experience in dem Spiel Gravity Warrior erklärt. 
Mithilfe der Gravitation, die von den Asteroiden ausgeht, kann der Spieler einfach und elegant von Asteroid zu Asteroid gelangen. Dabei muss er darauf achten, nicht mit herumfliegenden Aliens zusammenzustoßen, die mit der Zeit mehr werden. Mithilfe des Jet Packs kann der Spieler geschickt agieren und den Aliens ausweichen. Die Gravitation beeinflusst die Projektile der Waffe des Spielers, sodass der Spieler ein bis jetzt einzigartiges Erlebnis erfährt, während er auf die Gegner zielt und die Projektile abfeuert. Im Laufe des Spiels erhöht sich die Gegneranzahl je nach dem aktuellen Schwierigkeitsgrad, sodass der Spieler seine Treffsicherheit unter Beweis stellen und verbessern muss, um in eine höhere Wave zu gelangen. Dadurch erfährt der Spieler verschiedene Erfolgserlebnisse. 
Aufgrund der zunehmenden Anzahl der Aliens muss der Spieler geschickt mit den anderen Mitspielern kooperieren und kommunizieren, um die verschiedenen Waves zu schaffen. Dabei erlangt der Spieler ein Gemeinschaftsgefühl. Dieses wird noch weiter verstärkt, weil die Spieler im Falle eines Spielerverlustes diesen wiederbeleben können (siehe \autoref{subsec:mec} \nameref{subsec:mec}). Dies motiviert die Spieler, da sie danach wieder vollzählig gegen die Aliens kämpfen können. 

Allerdings ist die Wiederbelebung nicht immer erfolgreich bzw. gestaltet sich schwer, wenn ein Spieler in einem Alienschwarm gestorben ist. Die Aliens versuchen, die Spieler und den von den Spielern zu beschützenden Satelliten mit ihren Projektilen zu treffen, um die Lebensenergie der Spieler zu reduzieren. Dadurch entsteht bei den Spielern eine Spannung, die umso höher wird, desto niedriger die Lebensenergie sinkt. Ist diese aufgebraucht, stirbt der Spieler. Sobald die Spieler eine Wave erfolgreich geschafft haben, tritt die Erleichterung beim Spieler auf, da er den riskanten Situationen erfolgreich entkommen ist. Neben den eben genannten kooperativen Aspekten treten agonale Aspekte auf. Der Spieler möchte z. B. besser und mehr Aliens treffen als seine Mitspieler oder das Team möchte mehr Waves schaffen als andere Teams. Der agonale Aspekt könnte zukünftig noch weiter verstärkt werden, indem ein High-Score System eingeführt wird (siehe \autoref{subsec:improvements}: \nameref{subsec:improvements}). 

\newpage
\subsubsection{Fantasy}
\label{subsec:fan}
Eine militärische Forschungsgruppe bestehend aus zwei bis vier Mitgliedern wird von der Bodenstation “Mission Control” auf der Erde auf eine Mission ins Weltall geschickt, um eine bestimmte hochgravitationelle Asteroidenkonstellation im Universum zu untersuchen. Die Gravitation dieser Asteroiden ist allerdings so hoch, dass die Gruppe auf einen Asteroiden notlanden muss. Dabei wird ihr Raumschiff beschädigt und von der Gravitation stark angezogen, sodass die Gruppe die Asteroidenkonstellation nicht mehr verlassen können. Die Forscher können lediglich mit ihren Jet Packs von einem zum anderen Asteroiden gelangen. Die Forscher installieren einen Satelliten, damit „Mission Control“ die Gruppe orten und wieder abholen kann. Daraufhin entdecken Aliens, die auf diesen Asteroiden zu Hause sind, die Eindringlinge und greifen die Forschungsgruppe an. Die Gruppe muss so lange überleben, bis das Signal des Satelliten von der Bodenstation geortet wurde, damit diese aus dem Gebiet gerettet werden kann.

\begin{figure}[htp]
	\centering
	\includegraphics[width=.8\textwidth]{comic.png}
	\caption{Story als Comic}
	\label{fig:comic}
\end{figure}

\subsubsection{Mechanics}
\label{subsec:mec}
Nachfolgend sind die im Spiel realisierten Game Mechanics, sowie die darin enthaltenen Actions, Goals und Rules erläutert. Da es sich bei Gravity Warrior um einen interaktiven 2D Shooter handelt, ist der Functional Space (nach Schell) kontinuierlich. Gleiches gilt für den Action Space, da neben diskreten Actions, wie die Aktivierung des Boosts, auch kontinuierliche Actions, wie das Zielen mit der Waffe, verwendet werden. Zu den Objects des Spiels gehören neben den Gravity Warrior die Aliens, der Satellit und die Planeten, wobei deren Eigenschaften grundverschieden sind. Bei der Umsetzung der Aliens wurde eine Finite State Machine entwickelt, welche zur Verwaltung des Behaviors eines Aliens verwendet wurde. In der Implementierung der Alienart Assassin wurden fünf States umgesetzt, welche zusätzlich für die Animation des Aliens mitgenutzt wurden. \autoref{fig:state_machine} veranschaulicht die umgesetzte State Machine mit den jeweiligen Übergängen zwischen den States.

\begin{figure}[htp]
	\centering
	\includegraphics[width=.65\textwidth]{state_machine.png}
	\caption{State Machine Assassine}
	\label{fig:state_machine}
\end{figure}

\lbparagraph{Actions}
Actions beschreiben die im Spiel zum Einsatz gekommenen Funktionalitäten, die ein Spieler ausüben kann. autoref{tab:actions} listet die im Prototyp realisierten Actions auf und kategorisiert diese in Operative Actions, die Grundhandlungen, die ein Spieler vollziehen kann und Resultant Actions, welche erst durch den Einsatz und der Kombination aus Operative Actions entstehen.


\begin{table}[htp]
\centering
\begin{tabular}{|l|l|}
\hline
\textbf{Operative Actions}&\textbf{Resultants Actions} \\
\hline
Schießen&Schießen mit Gravitation \\
\hline
Boosten (Jet Pack)&Projektile von Planeten abprallen lassen\textbf{*}\\
\hline
Bewegen (x,y)&Ausweichen (Bewegen \& Boosten)\\
\hline
Zielen (x,y)&Kettenreaktion\textbf{**} (Schießen auf Exploder)\\
\hline
Wiederbeleben&\\
\hline
\end{tabular}
\caption{Operative und Resultant Actions}
\label{tab:actions}
\end{table}

\textbf{*}Projektile abprallen lassen: Projektile können auf Planeten abgefeuert werden und prallen abschließend von diesen ab. Dies ermöglicht es, Aliens zu treffen, welche andernfalls schwer zu erreichen wären.

\textbf{**}Kettenreaktion: Durch die Zerstörung eines Exploders wird eine Explosion ausgelöst, welche umliegenden Alien Schaden zufügt. Befindet sich ein weiterer Exploder mit niedrigen Lebenspunkten im Umfeld der Explosion, löst dieser ebenfalls eine Explosion aus, was das Ausmaß des Schadens vergrößert. 


\lbparagraph{Rules}
Die Operational Rules des Spiels beschreiben, wie sich ein Spieler im Spiel zu verhalten hat\cite{K2011}. In Gravity Warrior entspricht das der Grundlage, dass die Spieler das Spiel erst durchgespielt haben, sobald der Satellit das Notsignal lang genug senden konnte und die Rettung erfolgreich durchgeführt wurde. Dabei müssen mehrere Wellen von Gegenspielern überlebt werden, ohne dass alle Gegenspieler kampfunfähig werden oder der Satellit zerstört wird. Ein Spieler gilt als kampfunfähig, sobald seine Lebensanzeige leer ist. Gelingt es nicht, den Satelliten vor einer Zerstörung zu schützen, oder wird das gesamte Team von den Aliens besiegt, gilt das Spiel als verloren. Um eine Welle an Aliens zu besiegen und in die Buff-Auswahl zu gelangen müssen alle Aliens dieser Welle vernichtet werden. Um das Spiel zu gewinnen, ist es notwendig, dass alle Gegnerwellen erfolgreich bewältigt wurden.

Foundational Rules hingegen spezifizieren die grundlegenden formalen Strukturen, die festlegen, wie das Spiel funktioniert\cite{K2011}. Dazu zählt die fest definierte Gravitation der einzelnen Asteroiden, welche aufgrund ihrer verschiedenen Größen unterschiedlich stark auf die Spieler und deren Projektile wirken, sowie die Verhaltensweisen der Aliens und deren Änderungen, welche beim Unterschreiten eines definierten Abstands zwischen Spieler und Alien eintreten. Des Weiteren spielt der eingestellte Schwierigkeitsgrad des Spiels und die daraus resultierende Anpassung des Spawnverhaltens und der maximalen Schadenspunkte der Aliens hier eine Rolle.
\lbparagraph{Goals}
Goals definieren in Spielen die wichtigsten Rules, da diese festlegen, was der Spieler für ein erfolgreiches Bestreiten des Spiels benötigt\cite[~p.145]{S2014}. Im Spiel gilt es die folgenden goal types zu unterscheiden: \\
\textbf{Short term:} Treffen und besiegen eines Aliens, Angriffen eines Aliens ausweichen, Lebenspunkte regenerieren, Mitspieler wiederbeleben \\
\textbf{Mid term:} Gegnerwelle abgeschlossen, neue Fähigkeit für Warrior auswählen \\
\textbf{Long term:} Spiel gewinnen, Spiel in fordernden Schwierigkeitsgraden bewältigen, Spezialisierung der Fähigkeiten eines Spielers

\subsubsection{Aesthetics}
Aufgrund des zum Anfang festgelegten Schwerpunktes auf der Gravitation der Asteroiden und der daraus resultierenden Wirkung auf den Spieler wurde es als passend beurteilt das Spiel in das Setting Weltall zu verlegen. Mithilfe der Leere und Dunkelheit der Spielwelt war es möglich, den Fokus des Spielers stärker auf die hellen Farben der zu steuernden Warrior und deren Projektile zu lenken. Des Weiteren wurde der Boost durch einen Traileffekt\footnote{Traileffekt: Ein visueller Effekt der einen Schweif hinter einem sich bewegenden Objekt erzeugt.} visualisiert um dem Spieler den temporären Geschwindigkeitsgewinn besser zu veranschaulichen. Anhand der grünen Lebensleiste des Satelliten sollte dem Spieler die Zugehörigkeit von diesem zur eigenen Mannschaft symbolisiert werden. Dasselbe Prinzip wurde bei der Regenerationszone des Satelliten angewandt, wobei diese durch die sinusähnliche Ausbreitung dem Spieler einen Bereich der Sicherheit vermitteln sollte. 
Obwohl die ersten Konzepte der Aliensprites noch detailliert umgesetzt waren, wurde sich im späteren Verlauf der Entwicklung darauf geeinigt, das gesamte Spiel in einem minimalistischen Kunststil umzusetzen. Dies wurde aufgrund der Weite der Spielwelt und dem daraus resultierenden sichtbaren Detailgrad für den Spieler festgelegt.
Durch die Einbindung von Audiofeedback bei der Betätigung des Boosts, dem Treffen eines Aliens oder dem Verlust von eigenen Lebenspunkten konnte das Spielgefühl erneut bestärkt werden. Feinheiten wie die Heartbeat Vibration bei zu niedrigem Leben oder das leichte Zerteilen der Aliens bei deren Zerstörung wurde meist nur unterschwellig wahrgenommen. Diese wurden abschließend jedoch noch hinzugefügt, um dem Spiel einen höheren Detailgrad zu vermitteln.

\subsubsection{Technology}
\label{subsec:tech}
Der Kernbereich Technology im Elemental Tetrad befasst sich mit der Nutzung und Einbindung der Technologie in das Spiel, um eine gezielte Experience für den Spieler zu erzeugen\cite[~p.412]{S2014}. In Gravity Warrior wurde dabei die Verwendung von Gamepads eingebunden und als Spielvoraussetzung festgelegt, um das Gefühl von Couch-Coop besser zu vermitteln. Dies wurde durch die Variable Spieleranzahl unterstützt, wobei immer die Anzahl der verbundenen Controller, der Anzahl der am Spiel teilnehmenden Spielern entspricht (bis zum Maximum von vier Spielern). Des Weiteren wurde sich gegen die Einbindung eines Online-Multiplayer entschieden, da dieser in Gravity Warrior eher als Hürde für die direkte Kommunikation und das Teamplay zwischen den Spielern wirkt. Die Verbindung aus einem physischen Treffen und Zusammensitzen wurde präferiert, um somit gegebenenfalls das, aus der Vergangenheit bekannte Gefühl der Splitscreen und Couch-Coop Spiele, zu vermitteln.

Die verwendete Game Engine Godot, welche bei der Entwicklung zum Einsatz kam, wurde aufgrund ihrer einfachen Nutzung und der kostenlosen Bereitstellung ausgewählt. Zusätzlich ist die Engine als Open Source deklariert, wobei sämtlicher Quellcode durch die Nutzer eingesehen und angepasst werden kann. Die Möglichkeit eines Deployment zu Konsolen ist ebenfalls durch Godot gegeben. Hauptvorteil der Engine liegt in der seperaten Umsetzung der Engine im Bereich 2D und 3D, da Godot anders als Unity eine komplett unabhängige 2D Engine bereitstellt und somit keine Projektion der 3D Engine ist\cite{G2019}.

\newpage
\section{Entwicklungsprozess}
Im Folgenden wird der zeitliche Verlauf der Ausarbeitung der Spielidee, sowie dessen Umsetzung, beschrieben. Dabei werden vor allem Herausforderungen, die während des Prozesses auftraten thematisiert, sowie die dafür entwickelten Lösungen.
\subsection{Ideenfindung}
Zu Beginn des Projektes wurden die verschiedenen Projektideen gesichtet und auf ihre Umsetzbarkeit überprüft. In einer ersten Ideenfindung entstanden drei grundsätzliche Vorschläge, die zunächst vage beschrieben wurden.

\lbparagraph{Omega}
Ein Kartenspiel, bei dem die Technologie Mobile und PC kombiniert werden sollte. Karten sollten über das Handy ausgespielt und aktiviert werden können, während der allgemeine Spielstatus auf dem PC angezeigt wird. Eine genaue Vorstellung (Ziel, Regeln, Thema) des Spiels existierte noch nicht.
\lbparagraph{Attack on Human}
Ein Strategiespiel dessen Ziel es ist eine Körperzelle einzunehmen, indem verschiedene Angreifer (Viren/Bakterien) kontrolliert werden. Dazu muss das Verhalten der Zelle verstanden und ausgenutzt werden.

\lbparagraph{Gravity Warrior}
Ein 2D Multiplayer Spiel, bei dem jeder Spieler einen bewaffneten Avatar durch ein Asteroidensystem steuert, während er Gegner bekämpft. Kernidee ist, dass der Avatar, sowie dessen Schüsse, von der Gravitation der umliegenden Asteroiden beeinflusst wird.
Es wurden verschiedene Vorschläge bezüglich des Spielzieles abgewägt, wie ein PvP Spiel, bei dem Mitspieler abgeschossen werden mussten oder eine Verteidigung gegen angreifende Außerirdische. Da Gravity Warrior am ausgereiftesten erschien und von der Idee am meisten überzeugte, wurde sich abschließend dafür entschieden, dieses Spiel umzusetzen.

\subsection{Erster Prototyp des Player Movement}
Ziel der ersten Iteration des Entwicklungsprozesses war ein Prototyp, in dem es möglich war, seinen Gravity Warrior durch eine Menge von Planeten zu steuern. Auf welche Art und Weise die Gravitation der Planeten auf den Gravity Warrior wirkte sowie das Movement des Spielers auf einem Planeten wurde dabei primär betrachtet. Diskutiert wurde vor allem, ob für die Situation auf dem Planeten ein anderes Movement implementiert werden sollte, als für die Situation im freien Raum. Ein Vorschlag war es den Spieler entweder im Uhrzeigersinn oder gegen den Uhrzeigersinn, um den Planeten zu bewegen, abhängig davon, ob der Spieler die A oder die D Taste betätigt. Schlussendlich wurde sich dafür entschieden das Movement auf dem Planeten ähnlich dem Movement im freien Raum zu gestalten, da sonst die Umstellung zwischen den beiden Bewegungsarten zu überraschend und kompliziert wäre.

\subsection{Das Konzept des Spiels}
Eine der ersten Entscheidungen, die das Konzept des Spiels betrafen, war, dass Gravity Warrior als ein Couch-Coop Spiel entworfen werden sollte.
Zu diesem Zeitpunkt stand das Ziel des Spiels, das die Spieler erreichen sollten, noch nicht fest. Es wurde sich dazu entschieden,ein Wave Defence Spiel gegen angreifende Aliens umzusetzen, nachdem auch andere Spieltypen, wie kompetitives Player vs. Player oder das Angreifen einer gegnerischen Basis diskutiert wurden.
Passend zu diesem Spielprinzip wurde die in \autoref{subsec:fan} beschriebene \nameref{subsec:fan} konzipiert, die den Spielern die Situation der Gravity Warrior vermittelt und gleichzeitig den Kernaspekt Gravitation ins Zentrum rückt.
Um das neue Spielkonzept zu testen, wurde eine erste Variante von angreifenden Aliens implementiert, die an zufälligen Positionen am Rand des Bildschirms erscheinen, um anschließend auf den Spieler zu zufliegen. Asteroiden werden auf dem Weg zum Spieler ignoriert. Berühren die Aliens den Spieler, so werden dem Spieler Leben abgezogen. Um dies zu visualisieren, wurde eine Lebensanzeige für den Spieler entwickelt, die ebenfalls den Booststatus veranschaulicht.
Der Spieler bekam die Möglichkeit zu schießen und wurde ein Alien getroffen, so verschwand es augenblicklich.


\subsection{Erweiterung der Features}
\lbparagraph{Weitere Alienarten}
Beim Testen des Prototypen wurde das Problem festgestellt, dass ein Alien den Spieler durch den Asteroiden hindurch angreifen konnte, während Schüsse vom Asteroiden abprallten und das Alien somit verfehlten. Um dieses Problem zu lösen, wurde mit neuem unterschiedlichen Verhalten von Aliens experimentiert und dieses iterativ verbessert.
Des Weiteren stellte es sich als problematisch dar, dass Aliens, die sich auf einem Asteroiden versteckten, um auf vorbeifliegende Spieler zu lauern, diese durch Projektile nur mäßig bis gar nicht zu treffen waren. Deshalb wurde ein Pathfinding Verfahren umgesetzt, sowie eine Angriffsbewegung, die ausgelöst wird, sobald sich ein Spieler in der Reichweite des Aliens befindet. Abschließend wurden weitere Aliens umgesetzt, die den neu entwickelten Satelliten angreifen. Im Verlauf der Entwicklung wurden auf Grundlage der bereits geschriebenen Basis noch weitere Abwandlungen der Aliens entwickelt.

\lbparagraph{Buff-Selection}
Durch die Buff-Selection konnten die Spieler am Ende einer Wave ihre Eigenschaften aufwerten. Eine Möglichkeit des Designs der Buff-Selection sah vor, dass alle Spieler über die zur Verfügung stehenden Buffs abstimmen, um dann den gewählten Buff zu erhalten. Die andere Variante funktionierte, indem jeder Spieler separat entscheiden konnte, welchen Buff er erhalten möchte. Die erste Annahme war, dass durch eine einheitliche Buff-Selection die Kommunikation zwischen den Spielern gefördert werden kann, weshalb dies initial präferiert wurde.
Später kam es jedoch zur Änderung in individuelle Buffs, da sich die Spieler so besser spezialisieren können, was das Teamwork ebenfalls fördert und so Uneinigkeiten vermieden werden konnten.

\lbparagraph{Waffentypen}
Zu Beginn des Spiels wurden Waffen mit unterschiedlichen Eigenschaften zufällig zwischen den Spielern aufgeteilt. Implementiert waren eine Machine Gun, eine Gatling Gun, die sich durch ihre hohe Schussrate auszeichnete, eine Rifle, was der vorherigen Standardwaffe entsprach und ein Launcher, der in einem Bereich um den Einschlagspunkt Schaden verursachen sollte. Der Launcher wurde aufgrund seiner unterschiedlichen Einsatzmöglichkeit und dem daraus resultierenden Entwicklungsaufwand nicht fertig implementiert.

\lbparagraph{Eigenschaften Asteroiden}
Die weiteren Features, die in diesem Iterationsschritt hinzukamen, waren die Eigenschaften der Asteroiden, die über die Farbe des Asteroiden angezeigt wurden. Die Eigenschaften Heilung der Spieler, verletzen der Spieler, Abstoßung statt Gravitation und Verlangsamen der Spieler wurden implementiert und wurden zu Beginn zufällig verteilt.

\lbparagraph{Reduktion der Features}
Wie sich beim Testen zeigte, wurde die Übersichtlichkeit durch die Fülle der neuen Features eingeschränkt, was den Reiz am Spiel minderte. Aus diesem Grund wurde sich dazu entschieden, die Eigenschaften der Asteroiden sowie die verschiedenen Waffentypen wieder aus dem Spiel zu nehmen.

\subsection{Abstimmung der Spielelemente}
Als einer der nächsten Schritte wurde ein System umgesetzt, das es erlaubte die Angriffswellen der Gegner mit zunehmender Intensität angreifen zu lassen. Damit einhergehend wurde auch ein Ziel manifestiert, das darin bestand die letzte gegnerische Welle zu überstehen. Um den Sieg beziehungsweise die Niederlage anzuzeigen, wurde ein Win-Screen sowie ein Lose-Screen erstellt und ins Spiel integriert.
Anschließendes Testen mit mehreren Spielern zeigte dann, dass bei vier Spielern die Übersichtlichkeit durch die Menge der gegnerischen Einheiten beeinträchtigt wurde. Um diesem Problem entgegenzuwirken, wurden die einzelnen Einheiten verstärkt und die Anzahl reduziert, um somit trotzdem ein forderndes Spielerlebnis zu schaffen.
Das größte Problem, dass während dieser Phase auftrat, war, dass die Gravitation, die namensgebend und zentraler Bestandteil der Spielidee ist, den Spieler eher behinderte, anstatt das Spielerlebnis zu bereichern. Vor allem die Bewegungen des Spielers wurden durch die Gravitation gestört. Abschließend erfolgte die Einigung die Gravitation auf den Spielern relativ gering zu halten, während die Projektile nach wie vor stark von der Gravitation manipuliert wurden.

\subsection{Verbesserung der Experience}
Um die Experience zu verbessern und den Spieler besser über die aktuelle Spielsituation zu informieren wurden Animationen, Sounds und Controller-Feedback in das Spiel integriert. Geachtet wurde besonders darauf, dass wichtige Geschehnisse transportiert werden, wie das Treffen oder Töten eines Gegners, das von einem Gegner Getroffen-Werden, sowie ein geringer Lebensstand oder das Sterben eines Spielkameraden. Des Weiteren wurde mittels einer abstoßenden Kraft verhindert, dass ein Spieler in den Bereich außerhalb des Bildschirms fliegen kann. Um auf anbahnende, noch nicht vollständig sichtbare Aliens aufmerksam zu werden, wurden zusätzlich kleine Icons als Indikatoren am Spielfeldrand hinzugefügt.
Für die Anpassung des Schwierigkeitsgrades wurden abschließend multiple Runden gespielt, wobei dieser dabei iterativ verändert wurde. Herausfordernd an dieser Stelle war, dass die ersten Waves einen guten Einstieg für neue Spieler bereitstellen sollten, während sie für erfahrene Spieler nicht zu langweilig erscheinen durften. Die Ideen, die dieses Problem lösten, waren ein Regler, mit dem die Spieler vor Spielstart den Schwierigkeitsgrad der Angriffswellen bestimmen konnten sowie Spieletests mit deren Hilfe die allgemeine Schwierigkeit angepasst wurde.

\newpage
\section{Fazit}
\subsection{Erweiterungsmöglichkeiten}
\label{subsec:improvements}
\newpage
\section{Anhang}
\label{sec:appendix}
\subsection{User Tests}
\subsection{Repository}
\bibliographystyle{plain}
\bibliography{references}
\end{document}
